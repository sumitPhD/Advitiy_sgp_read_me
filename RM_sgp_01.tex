\documentclass[a4paper, oneside,11pt]{article}
\usepackage[a4paper,top=3cm,bottom=3cm,left=3cm,right=3cm,marginparwidth=1.75cm]{geometry}
\usepackage[utf8]{inputenc}
\usepackage{lipsum}
\usepackage{graphicx}
\usepackage{times}
\usepackage{xcolor}
%\usepackage{booktabs}
\usepackage{float}
\usepackage{color}
\usepackage{hyperref}
\usepackage{amsmath}
%\usepackage{subfig}
\usepackage{subcaption}
\usepackage[backend=biber,
    style=numeric,sorting=none]{biblatex}
\addbibresource{ref.bib}
\hypersetup{
    colorlinks=true,
    linkcolor=blue,
    urlcolor=blue,
}
\graphicspath{{/}}
\usepackage{adjustbox}
\usepackage{tabularx}
\usepackage{multirow}
\usepackage{titlesec}
\usepackage{graphicx}
\usepackage{upgreek}
\usepackage{enumitem}  
\usepackage{comment} 
\usepackage{upgreek}
%%%%%%%%%%%%%%%%%%%%%%%%%%%%%%%%%%%%%%%%%%%%%%%%%%%%%%%%%%%%%%%%%%%%

\begin{document}

\begin{table}[h]
		\begin{adjustbox}{width = \linewidth}
			\begin{tabular}{c c c}
				\multirow{5}{*}{ \includegraphics[width=0.13\textwidth]{iitb_logo.png}} \hfill &  \large{{Student Satellite Project}}  & \hfill \multirow{5}{*}{ \includegraphics[width=0.13\textwidth]{pratham_logo.png}} \\
				& {Indian Institute of Technology, Bombay} &\\
				& {Powai, Mumbai - 400076, INDIA} &\\
				&{} &\\
				& Website: {www.aero.iitb.ac.in/satlab} &\\
				\\
				&\large{\textbf{Readme file for sgp.py}}&\\
				&Attitude Determination and Control Subsystem&\\
				\hline
			\end{tabular}
		\end{adjustbox}
\end{table}
\section*{}
\textbf{Author:Sumit Agrawal}\\
\textbf{Date:23 July 2018}\\
sgp.py takes following input from getorbitdata.py and given postion and velocity of satellite in ECI frame in m and m/s. \\
Input:MeanMo - Mean motion in revolution per day\\
Eccen - Eccentricity\\
Incl\_deg - Inclination in degrees \\
MeanAnamoly\_deg - Mean Anamoly in degrees \\
ArgP - Argument of perigee in degrees \\
RAAN\_deg - Right ascension of ascending node in degrees\\
DMeanMotion - First time derivation of mean motion divided by 2 \\
DDMeanMotion - Second time derivation of mean motion divided by 6 \\
BStar - BSTAR drag term\\
Output: Postion and velocity of satellite in ECI frame in m and m/s. \\

References:


\printbibliography 
\end{document}
